\chapter{CSXCAD Geometrical Setup}\label{chap-CSXCAD}
	This chapter describes about the setup of properties and primitives of model using the \texttt{CSXCAD} interface. \texttt{CSXCAD} implies a C++ library to describe object geometries and their physical or non-physical properties in an open-source .xml format.  It provides a flexible and mesh-independent geometry definition and supports rectilinear mesh grid. OpenEMS gives absolute liberality to user in aspect of model geometry, by describing it with some specified functions in in Matlab platform (.m file). 
 
\textcolor{blue}{\begin{large}\textbf{InitCSX}\end{large}}\\
In order to initiate a structure description, one should firstly choose either a Cartesian or Cylindrical coordinate based mesh definition with function \hyperref[para:InitCSX]{\texttt{InitCSX}}.

\textcolor{blue}{\begin{large}Syntax:\end{large}}
 \begin{lstlisting}
CSX =InitCSX();
 \end{lstlisting}
 
\textcolor{blue}{\begin{large}Description:\end{large}}\\
\hyperref[para:CSX]{\texttt{CSX}}
  \begin{myindentpar}\hyperref[para:CSX]{\texttt{CSX}} describes geometrical objects and their physical or non-physical properties in .xml file. This syntax defines a default cartesian mesh. If argument \textcolor{green} {'CoordSystem','1'} is added, the cylindrical mesh definition is set. 
   \end{myindentpar} 
   
   
\begin{lstlisting}
CSX = InitCSX('CoordSystem','1');
\end{lstlisting}
 
   

\section{CSXProperties}\label{csx_prop} 
CSProperties describe the physical and non-physical properties of a model and has to be defined before the primitive properties. 

\subsection{Physical CSXProperties}
The physical CSXProperties which can be used in OpenEMS: 
 
\begin{enumerate}
\item Material Properties
\item Metal
\item Discretized material definition 
\item Dispersive Material
\item LumpedElement
\end{enumerate}

\subsubsection{Material Properties}

\textcolor{blue}{\begin{large}\textbf{AddMaterial}\end{large}}\\
The {\hyperref[para:Material Properties ]{\texttt{Material Properties }}} such as relative conductivity, permittivity and permeability can be defined with function \hyperref[para:SetMaterialProperty]{\texttt{SetMaterialProperty}}. Prio to it, the specified material has to added into CSX with function \hyperref[para:AddMaterial]{\texttt{AddMaterial}}.\\

\textcolor{blue}{\begin{large}Syntax:\end{large}}
 \begin{lstlisting}
CSX = AddMaterial(CSX, name,varargin)
 \end{lstlisting}
 
 \textcolor{blue}{\begin{large}Description:\end{large}}\\
  This function adds a material property to CSX with the given name.
  
 \hyperref[para:CSX]{\texttt{CSX}} 
 \begin{myindentpar}\hyperref[para:CSX]{\texttt{CSX}} describes geometrical objects and their physical or non-physical properties in .xml file.
 \end{myindentpar} 
 \texttt{name} 
 \begin{myindentpar}{\texttt{name}} is given by user and must be matched with that is mentioned in the syntax 
 \hyperref[para:SetMaterialProperty]{\texttt{SetMaterialProperty}}.
 \end{myindentpar} 
 \texttt{varagin} 
\begin{myindentpar} 
  \textcolor{green}{Isotropy} :If it is set to be '0', then an anistropy object is defined. 
\end{myindentpar} 


\textcolor{blue}{\begin{large}\textbf{SetMaterialProperty}\end{large}}\\
 \textcolor{blue}{\begin{large}Syntax:\end{large}}
  \begin{lstlisting}
 CSX = SetMaterialProperty(CSX, name, varargin)
  \end{lstlisting}
  \hyperref[para:CSX]{\texttt{CSX}} 
 \begin{myindentpar}\hyperref[para:CSX]{\texttt{CSX}} describes here the object physical properties in .xml file.
 \end{myindentpar} 
 \texttt{name} 
 \begin{myindentpar}{\texttt{name}} is given by user and must be matched with that is mentioned in the syntax 
 \hyperref[para:AddMaterial]{\texttt{AddMaterial}}.
 \end{myindentpar} 
 \texttt{varargin}
 \begin{myindentpar} Varargin is the argument has to be defined by user to specific the{\hyperref[para:Material Properties ]{\texttt{Material Properties }}}. 
  \begin{itemize}
  \item \textcolor{green}{Epsilon} :Relative permittivity [1 by default]
   \item \textcolor{green}{Mue}     :Relative magnetic permeability[1 by default]
  \item \textcolor{green}{Kappa}   :Electric conductivity[$\frac{S}{m}$]
  \item \textcolor{green}{Sigma}   :Magnetic conductivity [$\frac{\Omega}{m}$]
  \item \textcolor{green}{Density} :Material mass density[$\frac{Kg}{m^{3}}$] 
  \end{itemize}
 \end{myindentpar}


\textcolor{blue}{\begin{large}\textbf{SetMaterialWeight}\end{large}}\\

\textcolor{blue}{\begin{large}Syntax:\end{large}}

  \begin{lstlisting}
 CSX = SetMaterialWeight(CSX, name, varargin)
  \end{lstlisting}
  
\textcolor{blue}{\begin{large}Description:\end{large}}\\
It weights a material property with given weighting function by using the variables.
 \hyperref[para:CSX]{\texttt{CSX}} 
 \begin{myindentpar}\hyperref[para:CSX]{\texttt{CSX}} describes geometrical objects and their physical or non-physical properties in .xml file.
 \end{myindentpar} 
 \texttt{name} 
 \begin{myindentpar}{\texttt{name}} is given by user and must be matched with that is mentioned in the syntax 
 \hyperref[para:SetMaterialProperty]{\texttt{SetMaterialProperty}}.
 \end{myindentpar} 
 \texttt{varagin} The following variables can be used to define the weighting function:
\begin{myindentpar} 
  \begin{itemize}
  \item \textcolor{green}{x},\textcolor{green}{y},\textcolor{green}{z} :the distance from x,y or z along its axis to the origin
  \item \textcolor{green}{rho}       : ???
  \item \textcolor{green}{r}         :the radial distance from point of interest to orign
  \item \textcolor{green}{a}         :the polar angle as in cylindrical and spherical coordinate systems
  \item \textcolor{green}{t}         :the azimuthal angle in spherical coordinate system 
  \end{itemize}
\end{myindentpar} 

\textcolor{blue}{\begin{large}Examples:\end{large}}\\

\begin{lstlisting}
 CSX = AddMaterial( CSX, 'RO3003' );
 CSX = SetMaterialProperty( CSX, 'RO3010', 'Epsilon', 3, 'Mue', 1 );
\end{lstlisting}
 
 The first syntax adds a dielectric material named Ro3003. It has relative permittivity of 3 and relative permeability of 1. If lossy dielectric is introduced, one can set \textcolor{green}{Kappa} ($\sigma$)to its correponding value with known lost tangent ($\delta$): \begin{equation}
\sigma=2*pi*freq*\varepsilon*\tan(\delta)
\end{equation} 
\begin{lstlisting}
 CSX = AddMaterial( CSX, 'RO3003' );
 CSX = SetMaterialProperty( CSX, 'RO3010', 'Epsilon', 3,'Kappa',\sigma );
\end{lstlisting}

The following example shows anisotropic material property: 

\begin{lstlisting} 
  CSX = AddMaterial( CSX, 'sheet','Isotropy',0);
  CSX = SetMaterialProperty(CSX, 'sheet', 'Kappa', [0 0 kappa]);
\end{lstlisting}

The anisotropic material named sheet has zero \textcolor{green}{Kappa} at x- and y- direction but values 'kappa' at z-direction, where 'kappa' is predefined. If cylindrical coordinate system has been used, then the material sheet has only value at z-direction but not at radial and azimuthal direction. 
    
\begin{lstlisting} 
CSX = AddMaterial(CSX, 'abc');
CSX = SetMaterialProperty(CSX, 'abc', 'Epsilon', 1);
CSX = SetMaterialWeight(CSX, 'abc', 'Epsilon', ['(sin(4*z / 1000 *2*pi)>0)+1']);
\end{lstlisting}

A material named 'abc' has been added into CSX.Its relative permittivity has been weighted with function sin depends on its z-position. 

\subsubsection{Metal}
 This section shows how a metal/PEC is introduced into CSX. 

\textcolor{blue}{\begin{large}\textbf{AddMetal}\end{large}}\\
\textcolor{blue}{\begin{large}Syntax:\end{large}}

  \begin{lstlisting}
 CSX = AddMetal(CSX, name)
  \end{lstlisting}
  
\textcolor{blue}{\begin{large}Description:\end{large}}\\
This function introduces perfect electric conductor of no loss into CSX.\\
\hyperref[para:CSX]{\texttt{CSX}} 
 \begin{myindentpar}\hyperref[para:CSX]{\texttt{CSX}} describes geometrical objects and their physical or non-physical properties in .xml file.
 \end{myindentpar} 
 \texttt{name} 
 \begin{myindentpar}{\texttt{name}} is given by user and must be matched with that is mentioned in the syntax for defining CSXprimitives~\ref{CSXprimitives}.  
 \end{myindentpar} 

 If a lossy conducting material is preferred,then the following function is recommended:  
 
\textcolor{blue}{\begin{large}\textbf{AddConductingSheet}\end{large}}\\

\textcolor{blue}{\begin{large}Syntax:\end{large}}

  \begin{lstlisting}
CSX = AddConductingSheet(CSX, name, conductivity, thickness)
  \end{lstlisting}
  
\textcolor{blue}{\begin{large}Description:\end{large}}\\ 
\hyperref[para:CSX]{\texttt{CSX}} 
 \begin{myindentpar}\hyperref[para:CSX]{\texttt{CSX}} describes geometrical objects and their physical or non-physical properties in .xml file.
 \end{myindentpar} 
 \texttt{name} 
 \begin{myindentpar}{\texttt{name}} is given by user and must be matched with that is mentioned in the syntax for defining CSXprimitives\ref{CSXprimitives}.  
 \end{myindentpar} 
 \texttt{conductivity} 
 \begin{myindentpar} It is given by user according to the conductivity of metal being used. The most frequent used metal is copper and its conductivity is 58e6$Sm^{-1}$.    
 \end{myindentpar}
 \texttt{thickness} 
 \begin{myindentpar} it defines the thickness of metal sheet.     
 \end{myindentpar}

\textcolor{blue}{\begin{large}Examples:\end{large}}\\  

\begin{lstlisting} 
CSX = AddMetal(CSX,'metal'); 
\end{lstlisting}

This syntax adds PEC material into CSX with the name 'metal'. 

\begin{lstlisting} 
CSX = AddConductingSheet(CSX,'copper',56e6,70e-6);
\end{lstlisting}

This syntax adds a conducting sheet of 70$\mu$m into CSX. The assigned metal is named copper and has conductivity of 58e6$Sm^{-1}$.  

 
 
 
 
 
 
\subsubsection{Discretized material definition} 
\subsubsection{Dispersive Material}
\subsubsection{LumpedElement}
 
\subsection{non-Physical CSXProperties}
\subsubsection{Discretized material definition} 
\subsubsection{Dispersive Material}
\subsubsection{LumpedElement} 

\input{chapter/SEC_CSXCAD_Setup/primitives} \label{CSXprimitives} 

